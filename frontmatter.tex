\frontmatter
\chapter{前言}

\section*{本书的特点}
本书的例子均遵循google的编码规范(参考附录),希望读者在阅读本书示例代码的同时能够
直观的感受google编码规范并逐步养成遵守编码规范的好习惯。

\section*{本书读者对象}

\section*{如何阅读本书}


\section*{本书的体例}
\subsection*{印刷约定}
\begin{tabular}{|l|l|l|}
    \hline
    字体 & 意义 & 示例 \\
    \hline
    \textsl{AbCd123}斜体 & 文件名、路径名、域名等 & ls -l \textsl{filename}\\
    \hline
    \textbf{AbCd123}加粗 & 在终端输入的命令等 & subaochen\_desktop\% \textbf{su} \\
    \hline
    \texttt{等宽字体} & 示例代码、代码片段等 & \texttt{publc class MyClass...} \\
    \hline
\end{tabular}

\subsection*{图形标识}

本书使用了如下的图形标识帮助读者更好的区分和了解知识点所在:

\bgroup
\def\arraystretch{2.5}%
\begin{tabular}{ll}
\raisebox{-.4\height}{\includegraphics[width=8ex]{imgs/frontmatter/note.png}} & 需要注意的知识点。\\ 
\raisebox{-.4\height}{\includegraphics[width=8ex]{imgs/frontmatter/tip.png}} & 工程实践中常用的技巧。\\ 
\raisebox{-.4\height}{\includegraphics[width=8ex]{imgs/frontmatter/warning.png}} & 容易出错的地方,需要特别小心。\\ 
\end{tabular}
\egroup

\section*{如何使用本书的示例代码}

\section*{联系笔者}
由于笔者的写作水平有限,本书难免存在不少的不妥之处,还请广大读者谅解并不吝指教。
您可以通过我的博客获得最新的消息:\url{http://soft.sdut.edu.cn/blog/subaochen},或者给
我发Email:subaochen@126.com。本书中的示例代码可以从\url{http://github.com/subaochen/
jsf-primefaces-tutorial}下载,也欢迎读者不吝指教,在github提交PR,或者直接发邮件讨论亦可。
您可以在本书的不同章节看到如何获得源代码的相关提示。

\section*{本书是如何写成的}
本书全部使用开源(Open Source)软件完成:
\begin{itemize}
    \item Linux,本书所有稿件和源代码均在Linux下完成。
    \item git,本书的写作过程全程通过git进行版本控制,也借助于git在办公室、家和旅程中实现文档的同步,收益良多。
    \item Lyx(\url{http://www.lyx.org}),优秀的Latex前端可视化工具,最新版本
        (本书写作时是2.2)配合xetex、CTex可以很好的支持中文处理。
    \item graphviz,灵活而强大的代码绘图工具,本书部分流程图是使用graphviz绘制的。
    \item Shutter(\url{http://shutter-project.org}),Linux下面优秀的截图工具。
    \item ArgoUML(\url{http://argouml.tigris.org}),优秀的开源UML工具。
    \item umbrello(\url{https://umbrello.kde.org}),优秀的开源UML工具,本书部分UML图是使用umbrello绘制的。
    \item Dia,优秀的开源绘制工具,堪称Linux下面写作绘图的”瑞士军刀“,本书的大部分
        流程图、框图和UML图都是用Dia绘制的。
    \item Inkscape,优秀的开源矢量图绘制工具,本书的大部分矢量图是用Inkscape绘制的。
    \item vym(\url{http://www.insilmaril.de/vym/}),思维导图绘制工具,本书所有思维导图都是使用vym绘制的。
\end{itemize}

\section*{关于版权}
本书遵循Apache Licence,您可以在遵循Apache Licence和尊重原作者的的前提下,
在自己的博客、电子书等以电子文档的形式引用本书的内容,引用时请保留出处,
或者遵循常规的参考文献引用方式。

在本书正式出版前,笔者以及本书的其他贡献者一致要求,您不能将本书内容用于
正式出版物,包括但不限于本书的文字、图片和示例代码。

\section*{致谢}
\iffalse
写作如探险,一路坎坷,战战兢兢,如履薄冰。有的章节经过了屡次的推翻重来,
整本书的组织结构也经过了数次调整。在作为讲义内部使用的时候,学生们也给出了
很多中肯的建议和意见,笔者在授课过程中也不断的对本书的内容反思和整理。从构思
到成书,前后两年的时间,最终决定作为一本开源的Java系列教程。
\fi

在写作本书过程中,笔者查阅了大量Java教材和网络资料,所引用或依据的精彩论述和案例
尽量在文中标出,以便读者参照和比较,同时对原作者表示深深的敬意和感谢!

也感谢所有为开源软件作出贡献的人们!二十年前,当年懵懂的笔者决定彻底拥抱开源软件时,
为了安装一个Linux系统曾经不眠不休两天两夜;回头望,开源软件蓬勃发展二十年,
值得欣慰,值得弹冠相庆!没有Linux,没有Latex,没有Lyx,没有Dia,这本书也
不可能如此顺利的完稿。致敬,Open Source!

\hfill 宿宝臣

\hfill 2017年7月

\hfill 于山东理工大学1号实验楼314\#
\mainmatter
